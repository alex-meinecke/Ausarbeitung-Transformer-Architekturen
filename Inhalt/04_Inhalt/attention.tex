\chapter{Die Funktionsweise von Self-Attention}

Transformer arbeiten mit dem zentralen Baustein Self-Attention.
Self-Attention ist ein Algorithmus der letztendlich Zusammenhänge zwischen Wörtern z.B. in einem Text aufzeigt.
Wörter werden in Form von Tokens verarbeitet, die ganze Wörter oder Wortteile sein können.
Jeder Token ist einzigartig und wird zunächst nur durch eine natürliche Zahl repräsentiert.

\section{Generierung von Embeddings}

Grundlage jedes Attention-Zyklus sind Eingabe-Tokens.  
Um die mathematische Vorgehensweise besser zu veranschaulichen, wird im Folgenden der Satz \enquote{Ich sitze auf der Bank} als Beispiel verarbeitet.
Jedes dieser Wörter ist ein eigener Token, der vor der Eingabe in den Attention-Zyklus vom Transformer übersetzt wurde:
[\enquote{Ich}, \enquote{sitze}, \dots, \enquote{Bank}].  
Diese Tokens könnten für das Modell wie folgt aussehen: [\enquote{243}, \enquote{645}, \dots, \enquote{316}].
Die Herausforderung für den Transformer besteht darin, aus dem Kontext der anderen Tokens zu erkennen, ob \enquote{Bank} eine Sitzbank oder das Finanzinstitut Bank bedeutet.

Jeder Token bildet ein Schlüssel-Werte-Paar.  
Der korrespondierende Wert hinter einem Token ist ein Vektor, der die Bedeutung eines Tokens hinsichtlich mehrerer Dimensionen beschreibt.  
Diese Vektoren sind aus Trainingsdaten des Transformer-Modells entstanden.

Im ersten Schritt des Attention-Zyklus wird jeder Token in den dazugehörigen Vektor übersetzt.  
Diese Vektoren werden in der Matrix $\mathbf{X}$ gespeichert, wobei jede Zeile einen Token repräsentiert.  
Dieser Prozess wird als \textbf{Embedding} bezeichnet.  
Jeder dieser Vektoren hat gemäß der Literatur mindestens 512 Dimensionen.  
Es wird von einem \( d_{\text{model}} = 512 \) gesprochen.  
Es gilt: Je größer das \( d_{\text{model}} \), desto präziser kann der Transformer die Zusammenhänge zwischen Tokens erkennen.

Wenn sich Tokens im Vektorraum nahe liegen, haben sie eher Gemeinsamkeiten im Vergleich zu Tokens, die weit auseinander liegen.  
Angenommen, es gäbe nur ein \( d_{\text{model}} = 2 \) für jeden Token, könnten diese zwei Dimensionen als Koordinaten genutzt werden, um Zusammenhänge visuell als Cluster in einem Koordinatensystem darzustellen.  
Hier wären beispielsweise die Tokens \enquote{Hund} und \enquote{Katze} nah beieinander.

Für das oben genannte Beispiel \enquote{Ich sitze auf der Bank} nehmen wir der Übersichtlichkeit halber ein \( d_{\text{model}} = 4 \).  
So ergibt sich eine Embedding-Matrix $\mathbf{X}$ mit 5 Zeilen für 5 Tokens und 4 Spalten für jeweils 4 Dimensionen:

\[
\centering
\mathbf{X} =
\begin{bmatrix}
0.4 & 0.8 & 1.5 & 1.6 \\
3.2 & 0.4 & 0.7 & 0.2 \\
0.6 & 0.9 & 1.2 & 0.5 \\
2.1 & 0.5 & 2.0 & 0.2 \\
0.7 & 2.4 & 0.1 & 0.9
\end{bmatrix}
\]

\section{Lineare Transformation in Query-, Key- und Value-Matrizen}

Die Embedding-Matrix $\mathbf{X}$ wird durch drei Gewichtungsmatrizen $\mathbf{W_Q}$, $\mathbf{W_K}$ und $\mathbf{W_V}$, die aus dem Training des Transformer-Modells stammen, in drei neue Matrizen transformiert:

\[
\mathbf{Q} = \mathbf{X} \cdot \mathbf{W_Q}
\]
\[
\mathbf{K} = \mathbf{X} \cdot \mathbf{W_K}
\]
\[
\mathbf{V} = \mathbf{X} \cdot \mathbf{W_V}
\]

Die drei Matrizen haben im Attention-Zyklus umgangssprachlich formuliert folgende Funktionen:

\begin{itemize}
    \item \textbf{Query-Matrix (\(\mathbf{Q}\))}: Was fragt ein Token?
    \item \textbf{Key-Matrix (\(\mathbf{K}\))}: Welche Tokens im Kontext antworten am besten auf die Frage?
    \item \textbf{Value-Matrix (\(\mathbf{V}\))}: Erlernten Informationen über ein Token.
\end{itemize}

Im Beispiel könnten die jeweiligen Zeilen der Matrizen \(\mathbf{Q}\), \(\mathbf{K}\), \(\mathbf{V}\) für das Token \enquote{Bank} folgendermaßen aussehen:

\[
\begin{aligned}
\math{Q}_{\text{Bank}} &= [1.0, 0.7, 0.9, 1.1], \quad 
\math{K}_{\text{Bank}} &= [0.8, 0.6, 1.0, 0.9], \quad 
\math{V}_{\text{Bank}} &= [0.9, 0.5, 0.7, 1.0]
\end{aligned}
\]

\section{Berechnung und Einbeziehung von Attention-Scores}

Um die Relevanz zwischen \(\math{Q}\) und \(\math{K}\) zu messen, wird jeweils das Skalarprodukt zwischen jedem Tokenvektor von \(\math{Q}_{\text{T}}\) und \(\math{K}_{\text{T}}\) gebildet.  
Also im Beispiel wird unter anderem der Tokenvektor \(\math{Q}_{\text{Bank}}\) mit jedem Tokenvektor \(\math{K}_{\text{T}}\) multipliziert.

\[
\begin{aligned}
\text{Score}_{\text{Bank,Ich}} &= [1.0, 0.7, 0.9, 1.1] \cdot [0.4, 0.5, 0.1, 0.3] \\
&= 0.4 + 0.35 + 0.09 + 0.33 &= 1.17 \\
\text{Score}_{\text{Bank,Sitze}} &= [1.0, 0.7, 0.9, 1.1] \cdot [0.9, 0.8, 0.75, 0.8] \\
&= 0.9 + 0.56 + 0.675 + 0.88 &= 3.015 \\
\vdots \\
\text{Scores}_{\text{Bank}} &= [1.17, 3.015, 2.92, 1.12, 2.98]
\end{aligned}
\]


Die berechneten Attentionscores müssen noch zwei Verfahren unterlaufen.
Einmal ist das die Fokussierung und Normalisierung der Attentionscores mit der \textbf{Softmax-Funktion}.
\[
\text{Softmax}(x_i) = \frac{\exp(x_i)}{\sum_{j} \exp(x_j)}
\]

\(\math{x}\) ist der jeweilige aktuell zu betrachtende Attention-Score-Vektor.  
\(i\) ist der aktuell zu betrachtende Werteindex in diesem Vektor.  
\(j\) ist die Gesamtanzahl an Werten in \(\math{x}\).

Bei der \textbf{Fokussierung} werden höhere Attention-Score-Werte zwischen \(\mathbf{Q}\) und \(\mathbf{K}\) exponentiell bevorzugt.  
Analog dazu werden niedrigere Attention-Score-Werte exponentiell nach unten bewertet.  
Bei der zweiten Aufgabe der Softmax-Funktion, der \textbf{Normalisierung}, werden die Attention-Score-Werte pro Score-Vektor in Wahrscheinlichkeiten zwischen \(0\) und \(1\) transformiert, wobei die Summe jedes Attention-Score-Vektors immer \(1\) ist.

Damit die Softmax-Funktion aber optimal funktionieren kann, müssen die Werte in den Attention-Score-Vektoren erst einmal dimensioniert werden.
Bei geläufigen Transformermodellen wird wie oben beschrieben, ein \( d_{\text{model}} \) von mindestens 512 verwendet.
Durch diese großen Dimensionen entehen bei der Berechnung von den Attention-Scores durch die Aufsummierung bei der Bildung des Skalarprodukte sehr große Werte.
Diese großen Werte sorgen dafür, dass die Softmax-Funktion viele Q-K-Beziehungen sehr hoch bewertet und so der Transformer nicht sich auf die tatsächlich vielversprechenden Verbindungen konzentrieren kann und so die Weiterverarbeitung ungenau wird.
Diese Werte fallen bei dem oben gerechneten Beispiel nicht auf, da hier nur mit einem \( d_{\text{model}} \) von vier gerechnet wird.

Um hohe Attention-Score-Werte zu normalisieren, werden die Attention-Score-Vektor-Werte durch \( \sqrt{d_{\text{model}}} \) geteilt und so für die Softmax-Funktion in einen stabilen Bereich gebracht. \\
So kann ein Transformer auch kleine Relevanzunterschiede in der Token-Beziehung berücksichtigen.
Hier beispielsweise für das Attention-Score-Array von \enquote{Bank}:

\[
\frac{\text{Scores}_{\text{Bank}}}{\sqrt{4}} = [0.585, 1.5075, 1.46, 0.56, 1.49]
\]

\[
\text{Softmax}\left(\frac{\text{Scores}_{\text{Bank}}}{\sqrt{4}}\right) = [0.107, 0.269, 0.256, 0.104, 0.264]
\]

Damit diese nun umgewandelten Attention-Score-Wahrscheinlichkeiten in den weiteren Verarbeitungsschritten berücksichtigt werden können, werden sie mit der \( V \)-Matrix multipliziert. 
Das zeigt dem Modell, zu wie viel Prozent der erlernten Informationen zu einem Token im nächsten Schritt einfließen.
Insgesamt sieht das Verfahren folgendermaßen aus:

\[
\text{Attention}(Q, K, V) = \operatorname{Softmax}\left(\frac{QK^T}{\sqrt{d_{\text{model}}}}\right) V
\]

%Also werden die finalen die Informationen für Bank \math{V}_{\text{Bank}} folgendermaßen gewichtet:

%\[
%\math{Z}_{\text{Bank}} = [0.107, 0.269, 0.256, 0.104, 0.264] \cdot [0.9, 0.5, 0.7, 1.0] = [0.9, 0.5, 0.7, 1.0]
%\]

\section{Multi-Head-Attention}

Um Transformer noch effizienter zu machen, wird das Attention-Prinzip in Form der \textbf{Multi-Head-Attention} angewendet. 
Dabei beschreibt jeder einzelne Kopf einen eigenen Attention-Zyklus, wie er oben beschrieben wurde.

Jeder Kopf verwendet hierbei unterschiedliche Gewichtungsmatrizen, um den Fokus bei der Analyse der Zusammenhänge zwischen Tokens auf verschiedene Schwerpunkte zu setzen. 
So kann beispielsweise ein Kopf die Tokens hinsichtlich syntaktischer Beziehungen analysieren, während ein anderer die semantischen Beziehungen betrachtet. 
Für jeden Attention-Kopf \( h \) werden dabei unterschiedliche Gewichtungsmatrizen \( \mathbf{W}_{Qh} \), \( \mathbf{W}_{Kh} \) und \( \mathbf{W}_{Vh} \) verwendet.

Viele Transformermodelle verwenden acht Attention-Köpfe, die parallel ausgeführt werden.
Nach der parallelen Verarbeitung werden die jeweiligen Ergebnisse der Köpfe zusammenfügt, auch konkatinieren genannt, und mit Hilfe einer weiteren Gewichtungsmatrix \( \mathbf{W}_{O} \) zusammengefasst.
\( \mathbf{W}_{O} \) ist hierbei die Matrix, welche die Ergebnisse der Attention-Köpfe nach unterschiedlich gewichtet.
Umgangssprachlich formuliert, bildet sie eine einheitliche Schlussfolgerung für den Transformer aus diesen Muli-Head-Attention-Zyklus.

