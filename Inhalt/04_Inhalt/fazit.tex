\chapter{Fazit}

Transformer-Architekturen haben die Landschaft des maschinellen Lernens, insbesondere im Bereich der Textverarbeitung, revolutioniert. 
Durch den Einsatz von Self-Attention-Mechanismen und Multi-Head-Attention-Schichten können Transformer den Kontext von Tokens effizienter und umfassender erfassen als frühere Architekturen wie RNNs. 
Die Parallelverarbeitung ermöglicht es diesen Modellen, große Textmengen nicht nur schneller, sondern auch mit einer höheren Genauigkeit zu analysieren.

Das Ziel dieser Arbeit war es, die grundlegenden Konzepte der Transformer-Architektur im Kontext der Textverarbeitung einem Leser ohne umfangreiches Vorwissen zugänglich zu machen. 
Dieses Ziel wurde durch eine systematische und praxisnahe Darstellung der Schlüsselkomponenten wie Scaled Dot-Product Attention, Multi-Head Attention, Positional Encodings und der Feed-Forward-Schichten erreicht. 
Die detaillierte Beschreibung des Encoder-Decoder-Aufbaus verdeutlichte die Flexibilität der Architektur, Texte sowohl zu analysieren als auch zu generieren.

Darüber hinaus wurde anhand eines Beispiels die Funktionsweise des Attention-Mechanismus veranschaulicht, um die theoretischen Konzepte für den Leser greifbar zu machen. 
Die Analyse hat aufgezeigt, wie Transformer-Modelle den Kontext eines Textes erschließen, um übereinstimmende Tokens effizient zu gewichten und damit eine bessere semantische Analyse zu ermöglichen.

Besonders der Encoder-Decoder-Aufbau stellt eine zentrale Innovation dar. 
Der Encoder abstrahiert die Eingabetokens, indem er semantische und syntaktische Zusammenhänge lernt, während der Decoder diese abstrahierten Repräsentationen nutzt, um präzise und kontextuell passende Ausgaben zu erzeugen. 
Diese Trennung der Aufgaben erlaubt es, Text nicht nur zu klassifizieren, sondern auch generative Aufgaben wie maschinelle Übersetzung oder Textzusammenfassungen effizient umzusetzen.

Zusammenfassend hat diese Arbeit die Grundlagen der Transformer-Architektur erfolgreich vermittelt und damit das Verständnis der technologischen Fortschritte in der Textverarbeitung vertieft.

